\section{Osnovno}

JavaScript je jezik, ki je integriran v neko okolje, kot je npr. HTML.
\begin{verbatim}
<script>
<!-- KODA -->
</script>
\end{verbatim}

\underline{Obmmočje delovanja js kode je:} 
\begin{verbatim}
<script src = "/..pot do datoteke/imedatoteke.js"><\script>
\end{verbatim}

Operator \textbf{console.log();} izpiše vneseno vrednost/spremenljivko v konzolo, ki je dostopna v brskalniku z ukazom Ctrl + Shift + J
\section{PRAVILA}
\begin{itemize}
	\item angleške črke
	\item desetiška števila
	\item podčrtaj (\_) ločuje besede
	\item začetek stavka ne sme biti število
	\item loči velike in male črke(je case-sensitive) 
	\item nedovoljena uporaba razerviranih izrazov/funkcij(\&, = ...)
	\item "navednice" označujejo dobesedno navajanje/znak
	\item primer:
\end{itemize}

\begin{verbatim}
console.log(a); <!-- Izpiše vrednost spremenljivke -->
console.log("a"); <!-- Izpiše znak a -->
\end{verbatim}
\section{Spremenljivke}

\begin{verbatim}
var = a;
\end{verbatim}
Z enačajem se definira vrednost ali izraz spremenljivki na desni strani(var). Definicija se \textbf{VEDNO} konča s podpičjem.\\

Operator "=" definira spremenljivko in ji priredi neko vrednost ali izraz.

\texttt{a = 31;}

Spremenljivki a priredimo vrednost 31.\\

\subsection{Tipi spremenljivk}\
\newline
Operator \textbf{typeof()} vrne tip spremenljivke/vrednosti.\


\subsection*{Številski(number)}
VREDNOSTI: 41, 2.15, Nan, infinity\\
primer:
\begin{verbatim}
console.log(typeof (13)) <!-- v konzoli se nam izpiše "number"-->
\end{verbatim}

\subsection*{Boolov(boolean)}
VREDNOSTI: TRUE, FALSE
primer:
\begin{verbatim}
console.log(typeof TRUE) <-- v konzoli se nam izpiše "boolean"-->
\end{verbatim}

\subsection*{Znakovni niz(string)}
VREDNOST: "jabolko"\ --> string
primer:
\begin{verbatim}
console.log(typeof ("jabolko")) <!-- v konzoli nam izpiše "string"-->
\end{verbatim}

\subsection*{Dedoločen tip(undefined)}
VREDNOST: undefined

\section{Izrazi}
Izraz je del kode, ki razreši vrednost. Lahko vrednost dodeli spremenljivki ali pa jo ima sam.\

Na primer: Izraz \texttt{x = 7} dodeli vrednost 7 spremenljivki x.\

\underline{Pomembna lastnost izrazov je} PREDNOST ali \textit{precedence} operatorjev med seboj. To pomeni, da program bere operatorje glede na njihovo prioriteto  oz. prednost.\

Na primer: V izrazu \texttt{a + b * c} ima operator \textbf{*} prednost pred operatorjem $+$, zato program najprej zmnoži števili b in c in nato sešteje vsoto s številom~a.\

Operatorji istega tipa/prednosti se pa izvajajo v vrstnem redu iz leve proti desni(\textit{associativity}).

\section{Operatorji}

\subsection*{Aritmetični}

\begin{description}[align=left,labelwidth=3cm]
	\item[$+, -$] Unarni/Binarni
	\item[$\times, \div$]
	\item[$\%$] Ostanek pri deljenju
\end{description}

\subsection*{Primerjalni}
Ti operatorji vračajo le $TRUE$ ali $FALSE$.\

\begin{description}[align=left,labelwidth=3cm]
	\item [$>, >=$]	Večje, večje ali enako(pomembno je zaporedje znakov!!)
	\item [$<, <=$]	Manjše, manjše ali enako
	\item [$==$] Je enako	
	\item [$!=$] Ni enako
\end{description} 

Prav tako velja tudi: \texttt{"5"~==~5 --> $TRUE$}

\subsection*{Logični}

Ti operatorji vračajo le $TRUE$ ali $FALSE$.\

\begin{description}[align=left,labelwidth=3cm]
	\item [$\&\&$] Logični IN
	\item [$||$] Logični ALI
	\item [$!$] Negacija
	\item [$=$] Priredilni operator --> spremenljivki na levi strani priredi vrednost na desni strani.
\end{description}

\underline{Prednost in red izvajanja:}\

\begin{description}[align=left,labelwidth=3cm]
	\item [Aritmetični] --> 
	\item [Primerjalni] -->
	\item [Logični] -->
	\item [Priredilni] <--
\end{description}

\underline{Bljižnice(shorthands):}\

\begin{labeling}{\texttt{x $=$ x $+$ izraz}}
	\item [\texttt{x $=$ x $+$ izraz}] => v spremenljivko x shranimo vsoto spremenljivke asdasdasdsadasdasdasdasd
	\item [\texttt{x $+=$ izraz}] => okrajšan zgornji stavek
	\item[\texttt{x $-+$ izraz}] spremenljivki x odštejemo vrednost izraza
	\item [\texttt{x$++$}] => spremenljivki x se vrednost poveča za 1
	\item[\texttt{x$--$}] spremenljivki x zmanjšamo vrednost za 1
\end{labeling}

\subsection*{Vejični}

Ima še nižjo prioriteto, kot priredilni operator

\subsubsection*{Pogojni}
\texttt{pogoj ? ce\_je\_true : ce\_je\_false}

V prvi del pred vprašajem se vnese pogoj in nato izraz, ki se prebere, če je pogoj izpolnjen, po dvopičjem pa sledi izraz, če pogoj ni izpolnjen.

Primer:\begin{verbatim}
var x = 3;
var y = 0;
x > 2 ? y = 1 : y = 2;
console.log(y);			--> konzola v tem primeru izpiše 1
\end{verbatim}

\subsection*{Primerjalni}
\underline{Črke:}
Črke primerja po abecedi, velike črke so pred malimi. Primerja se od prve do zadnje.


\section{Diagram poteka}

\begin{tikzpicture}[node distance = 2cm, auto]
% Place nodes
\node [block] (init) {stavek};
\node [decision, right of=init] (decide) {izraz};
\path [line] (init) -- (decide);

\end{tikzpicture}

Obstaja prazen stavek, ki vsebuje le podpičje. Podpičja so neobvezna v Javascriptu, a jih je vseeno dobro uporabljati.\
\newpage
Primer stavka:
\begin{verbatim}
{
	stavek1;
	stavek2;
	...
	stavekN;	
}
\end{verbatim}
\subsection*{Stavek if/else}\
\begin{verbatim}
if (pogoj) stavek1 else stavek2
\end{verbatim}
Po pogoju, sledi glede na rezultat pogoja \textbf{LE EN STAVEK!!} Za več kot en stavek, se uporabi zaviti oklepaj.\

\begin{tikzpicture}
\node[cloud, node distance = 2cm](vnos){vnos};

\node [decision, below of=vnos] (pogoj) {pogoj};
\node[block, left of=pogoj, node distance = 3cm](stavek2){stavek2};
\node[block, right of=pogoj, node distance = 3cm](stavek1){stavek1};
\node[cloud, below of=pogoj](izhod){rezultat};
\path [line] (pogoj) -- node {true}(stavek1);
\path [line] (pogoj) -- node {false}(stavek2);
\path[line](vnos) --(pogoj);
\path [line](stavek2)|-(izhod);
\path [line](stavek1)|-(izhod);
\end{tikzpicture}
\newpage
\underline{Primer stavkov:}
Računanje idealne teže s podanim podatkom o spolu in višini. Izvozi podatek idealne teže.
\begin{verbatim}
<script>

    var teza;
    var visina;
    var spol;

    spol = prompt("Vnesi spol (m/z)");

    if(spol == "m"){
        teza = 48 + (visina - 150) * 0.9;
    }
    else{
        teza = 43 + (visina - 150) * 0.7;    
    }
    console.log("Tvoja idealna teža je" + teza + "kg");
    
</script>
\end{verbatim}

\subsection*{Stavek for}

\texttt{for (start; pogoj; korak) stavek}

For stavek izvaja določen stavek, dokler je pogoj uresničen. V for stavek se vnese štartni parameter. Ta se preveri v pogoju in se spreminja po koraku.

\begin{tikzpicture}
\node[cloud](start){start};
\node[decision, below of=start, node distance = 3cm](pogoj){pogoj};
\node[block, below of=pogoj, node distance = 3cm](stavek){stavek};
\node[nothing,left of=pogoj](vmes);
\node[block, below of=stavek, node distance = 3cm](korak){korak};
\path[line](start)--(pogoj);
\path[line](pogoj)--(stavek);
\path[line](stavek)--(korak);
\path[line](korak) -| (vmes);
\path[line](vmes) -- (pogoj);

\end{tikzpicture}

Npr: začnemo s številom i, katerega vrednost je 0. Če hočemo stavek ponoviti 4-krat, potem bomo povečevali naš i do števila 3(štetje se začne s številom 0) s korakom i(i++).

\begin{verbatim}
for(var i = 0; i<4; i++) console.log("Zdaj se izvajam v " + i +"-tem krogu")
\end{verbatim}